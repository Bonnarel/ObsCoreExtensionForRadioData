\section{Science use cases and translation to ADQL queries  based on ObsCore Radio extension  }
\label{ADQLusecases}


\textit{ The  queries in subsections \ref{sec:s_resolution_max} and \ref{sec:s_fov_min}
can be run with s\_resolution\_min and s\_fov\_max instead of s\_resolution\_max and 
 s\_fov\_min respectively, to obtain results in which at least part of the dataset satisfy the condition.\\
For instance, consider datasets with a large observed bandwidth, like those taken with wide/ultrawide band receivers.
The condition s\_resolution\_max<d\_typical may not be satisfied. At the same time, the user may want to select those 
dataset where the desired condition on spatial resolution is matched at least partially  within the observed band.}

\textit{Sections from \ref{sec:instrument-mode} illustrate instrument and observing modes parameters. They must  be completed by science cases.}

%\subsection{Science case }
\subsection{Use case - s\_resolution\_min}
\label{sec:s_resolution_min}
\textit{Comparative study of low surface brightness extended emission surrounding compact sources, like for instance Virgo A.
In order to avoid washing out of diffuse emission the spatial resolution should not be too high.
Select any dataset map (raster or on-the-fly) with s\_resolution\_min larger than the desired spatial resolution of 1 arcmin.}
Show me all datasets satisfying: \\
I. Minimum spatial resolution > 0.017 deg \\
II Target Virgo A or position inside 15 arcmin from 187.7059308,+12.3911232 \\
III Scan mode is raster map or on-the-fly map

\begin{verbatim}
SELECT * FROM ivoa.obscore
NATURAL JOIN  ivoa.obscore-radio 
WHERE
s_resolution_min > 0.017 
 AND (target_name = 'Virgo A' OR
CONTAINS(POINT(s_ra, s_dec),CIRCLE,187.7059308,+12.3911232,0.25)) = 1)) 
AND  (scan_mode = 'raster map' OR scan_mode = 'on-the-fly map')
\end{verbatim}

\subsection{Use case - s\_resolution\_max}
\label{sec:s_resolution_max}
\textit{Morphological study of SNR IC443. Select any dataset obtained in map scanning mode (raster or on-the-fly) with s\_resolution\_max less than the desired spatial resolution of 1 arcmin.}

Show me all datasets satisfying: \\
I. Maximum spatial resolution < 0.017 deg \\
II Target IC443 or position inside 15 arcmin from 94.2500000,+22.5699997 \\

\begin{verbatim}
SELECT * FROM ivoa.obscore
NATURAL JOIN  ivoa.obscore-radio 
WHERE s_resolution_max < 0.017 
AND
 (target_name = 'IC443' OR
CONTAINS(POINT(s_ra,s_dec),CIRCLE(94.2500000,+22.5699997,0.25)) = 1)) 
AND (scan_mode = 'raster map' OR scan_mode = 'on-the-fly map')
\end{verbatim}

\subsection{Use case - s\_fov\_min - large field of views}
\label{sec:s_fov_min}
\textit{Select any dataset obtained in map scanning mode (raster or on-the-fly) with minimum field of view larger than 0.8 degree. For instance, investigate the region surrounding cluster Abell 194.}

Show me all datasets satisfying:\\
I. Minimum FOV > 0.8 deg \\
II Target name = Abell 194 or position inside 15 arcmin from 21.5054167, -1.3672221 \\
\begin{verbatim}
SELECT * FROM ivoa.obscore
NATURAL JOIN  ivoa.obscore-radio 
WHERE s_fov_min > 0.8 AND
(target_name = 'Abell 194' OR
CONTAINS(POINT(s_ra,s_dec),CIRCLE(21.5054167,-1.3672221,0.25)) = 1)) 
AND (scan_mode = 'raster map' OR scan_mode = 'on-the-fly map')
\end{verbatim}

\subsection{Use case - s\_fov\_min - extended target completly included}
\label{sec:s_fov_max}
\textit{Pictor A has a typical angular extension  of 8 arcminutes , select any image/cube completely containing the target}
Show me all datasets satisfying: \\
I. Target name = Pictor A \\
II. The circle defined by the minimum FOV of the dataset fully contains the circle delimiting Pictor A. \\
\begin{verbatim}
SELECT * FROM ivoa.obscore
NATURAL JOIN  ivoa.obscore-radio 
WHERE target_name = 'Pictor A' AND
CONTAINS(CIRCLE(79.9571789, -45.7788479,(8/60)/2),
CIRCLE(s_ra, s_dec, s_fov_min/2)) = 1)
\end{verbatim}

\subsection{Use case - dataproduct\_type}
\textit{Select all observations of calibrator source 3C48 with the Medicina Grueff radio telescope at frequencies in the range 20-21 GHz, to investigate its flux variability (light curve) in the years 2000-2023}
Show me all datasets satisfying: \\
I. Target name = 3C 48 \\
II. obs\_collection = ?INAF-Medicina, single dish?\\
III. Observed frequency in the range 20-21 GHz \\
IV. dataproduct\_type = spatial\_profile or scan\_mode contains map\\
V. 51544 < observation time (MJD) < 60309 \\

\begin{verbatim}
SELECT * FROM ivoa.obscore
NATURAL JOIN  ivoa.obscore-radio
WHERE obs_collection EQ 'INAF-Medicina, single dish' 
AND (em_min >=20000 AND em_max <= 21000) 
AND (dataproduct_type EQ 'spatial profile' OR scan_mode LIKE "*map*") 
AND  target_name = '3C48' 
AND  (t_min >= 51544 AND t_max <= 60309)
\end{verbatim}

\subsection{Use case - f\_resolution, frequency ranges}
\label{sec:FreqRanges}

\textit{Select any cube with spectral resolution better than 1 MHz and spectral range inside the 1 to 1.5 GHz band (around HI 21 cm line).}

Show me all datasets satisfying:\\
I. Product type is cube \\
II. spectral resolution better than 1 MHz \\
III. em\_min and em\_max such than frequency bounds are included in the 1GHz to 1.5GHz range \\
\begin{verbatim}
SELECT *, 299792458 / em_max as f_min,  299792458 / em_min as f_max 
FROM ivoa.obscore 
NATURAL JOIN  ivoa.obscore-radio
WHERE dataproduct_type = 'cube' AND
f_resolution < 1000000 AND
299792458 / em_max  > 1.0 e+9 AND
299792458 / em_min < 1.5e+9
\end{verbatim}

\subsection{Use case - frequency selection  for images }
\textit{Any image or cube observed at frequencies below 1 GHz.}\\ \\
Show me all datasets satisfying:\\
I. Product type is image or cube \\
II. higher spectral limit < 1Ghz
\begin{verbatim}
      SELECT * FROM ivoa.obscore
      WHERE (dataproduct_type = 'image' 
      OR  dataproduct_type = 'cube') 
      AND 299792458 / em_min  < 1e9
\end{verbatim}
% Mireille Louys: shall we also look for spectra with these frequency constrains ? 


 
% end section Alexandra's science case

% start of Mark's (VLBI) science cases

% These use NATURAL JOIN
% Should these restrict the dataproduct_type?
\subsection{Use case - high resolution data around FRB targets }

\textit{Give me high-resolution data on possible persistent radio sources within an arc second of FRB 121102:}

\begin{verbatim}
SELECT * FROM ivoa.obscore NATURAL JOIN ivoa.obscore-radio
WHERE CONTAINS(POINT(s_ra,s_dec),CIRCLE(82.99458,33.14794,0.0003)) = 1 
AND s_resolution_max < 0.001
\end{verbatim}

\subsection{Use case - reasonable fidelity}

\textit{Give me data on extended HI emission around the source 3C84 that can
be imaged with reasonable fidelity:}

\begin{verbatim}
SELECT * FROM ivoa.obscore 
NATURAL JOIN ivoa.obscore-radio
WHERE CONTAINS(POINT(s_ra,s_dec),CIRCLE(24.4212500,33.1588889,0.003)) = 1 
AND s_largest_angular_scale > 0.018 
AND uv_distribution_fill > 0.2 
AND uv_distribution_ecc < 0.75 
AND em_min < 0.21 
AND em_max > 0.21
\end{verbatim}

% end of Mark's (VLBI) science cases

\subsection{ use case - visibility data product and target object selection }
\textit{Any raw interferometry observation containing target Fornax cluster.}\\
\\
Show me all  observation identifiers satisfying:\\
I. DataType = visibility \\
II. Target Name = Fornax Cluster \\
\begin{verbatim}
      SELECT obs_id FROM ivoa.obscore
      WHERE dataproduct_type = 'visibility' 
      AND target_name = 'Fornax cluster'
\end{verbatim}


\subsection{Use case - maximum angular scale selection}
\textit{Any visibility dataset Within an arcec around FRB 121102  where  maximum angular scale is larger than 1 minute .}\\ \\
Show me all datasets satisfying:\\
I. Product type is visibility \\
II. Maximum angular scale  >  0.018 deg \\
III. Position within 1 arcsec of 82.99458 + 33.14794
\begin{verbatim}
      SELECT * FROM ivoa.obscore 
      NATURAL JOIN  ivoa.obscore-radio
      WHERE dataproduct_type = 'visibility' 
      AND s_largest_angular_scale_max  > 0.018 
      AND CONTAINS(POINT(s_ra,s_dec),CIRCLE(82.99458, 33.14794,0.0003)) = 1))
\end{verbatim}

\subsection{Use case - maximum angular scale selection with reasonable fidelity}
\textit{Any visibility data within an arcsec around FRB 121102 where  maximum angular scale is larger than 1 minute  and 
of good quality from the uv distribution point of view.}\\ \\
Show me all datasets satisfying:\\
I. Product type = visibility \\
II. Maximum angular scale  >  0.018 deg \\
III. uv distribution filling factor  > 0.7 \\
IV. uv distribtion factor eccentricity  < 0.1 \\
V. Position within 1 arcsec of 82.99458 + 33.14794
\begin{verbatim}
      SELECT * FROM ivoa.obscore 
      NATURAL JOIN  ivoa.obscore-radio
      WHERE dataproduct_type = 'visibilty' 
      AND s_largest_angular_scale_max > 0.018 
      AND uv_distribution_fill > 0.7 
      AND uv_distribution_ecc < 0.1  
      AND CONTAINS(POINT(s_ra,s_dec),CIRCLE(82.99458, 33.14794,0.0003)) =1))
\end{verbatim}

\subsection{Use case -  time selection }
\textit{Any timeseries obtained in beam forming mode with time resolution better than 0.1 second and time sample maximal duration smaller than 0.05 second. }\\ \\
Show me all datasets satisfying:\\
I. Instrument type = beam forming\\
II. Product type = time series  \\
III. time resolution < 0.1 s\\
IIV. maximal duration per sample  < 0.05 s \\

\begin{verbatim}
       SELECT * FROM ivoa.obscore 
      NATURAL JOIN  ivoa.obscore-radio
      WHERE instr_type = 'beamForming' 
      AND dataproduct_type = 'timeseries' 
      AND t_resolution < 0.1 AND
      AND t_exp_max < 0.05 
\end{verbatim}

\subsection{Use case -  instrument type and mode selection }
\label{sec:instrument-mode}
\textit{Any single dish  dataset  in raster mode around 3C 273.}\\ \\
Show me all datasets satisfying:\\
I. Instrument type is Single dish \\
II. Scan mode is raster \\
III. Target name = 3C 273 or \\
IV. position inside 3 arcmin around  	187.2779159404900 +02.0523882305500
\begin{verbatim}
      SELECT * FROM ivoa.obscore 
      NATURAL JOIN  ivoa.obscore-radio
      WHERE instr_type = 'SD' 
      AND scan_mode = 'raster' 
      AND (target_name = '3C 273' OR
       CONTAINS(POINT(s_ra,s_dec), CIRCLE(187.2779159404900, +02.0523882305500,0.05)) = 1)     
\end{verbatim}

% use case FB 
\subsection{Use case - instrument type and frequency selection }
\textit{Any single dish or beam forming dataset  with spectral range inside the 500 Mhz - 1Ghz band and in the Coma Cluster.}\\ \\
Show me all datasets satisfying:\\
I. Instrument type is Single dish or beam forming \\
II. lower spectral limit > 500 Mhz \\
III. higher spectral limit < 1Ghz \\
IV. position inside 18 arcmin around 194.93502 +27.91246

\begin{verbatim}
      SELECT * FROM ivoa.obscore 
      NATURAL JOIN  ivoa.obscore-radio
      WHERE (instr_type = 'SD' OR
      instr_type = 'beamForming') 
      AND f_min > 500 
      AND f_max < 1000 
      AND CONTAINS(POINT(s_ra,s_dec),CIRCLE(194.93502, +27.91246, 0.3) = 1     
\end{verbatim}


\subsection{Use case - instrument parameters selection }
\textit{Any interferometry data of good quality and significant spatial resolution from the instrumental point of view. }\\ \\
Show me all datasets satisfying:\\
I. Instrument type = visibility \\
II. number of antennae > 40 \\
III. antenna maximum distance in meters  > 5000 \\

\begin{verbatim}
      SELECT * FROM ivoa.obscore-radio
      WHERE instr_type = 'interferometry' 
      AND instr_tel_number > 40 
      AND instr_tel_max_dist > 5000 
\end{verbatim}



