\documentclass[11pt,a4paper]{ivoa}
\input tthdefs

\newcommand{\xtype}[1]{\texttt{#1}}

\usepackage{listings}
\lstloadlanguages{XML,SQL}
\lstset{flexiblecolumns=true,basicstyle=\ttfamily}
\usepackage[utf8]{inputenc}
\usepackage{todonotes}
\usepackage{textcomp}
\usepackage{float}
\usepackage{lscape}
\usepackage{longtable}
\hyphenation{image/mosaic}

\marginparwidth=4cm

\title{IVOA Obscore Extension for Radio data}

\ivoagroup{Data Model Working Group}

\author{Fran\c cois Bonnarel}
\author{Mireille Louys}
\author{Baptiste Cecconi}
\author{Vincenzo Galluzzi}
\author{Yan Grange}
\author{Mark Kettenis}
\author{Mark Lacy}
\author{Alan Loh}
\author{Mattia Mancini}
\author{Peter Teuben}
\author{Alessandra Zanichelli}




\editor{Fran\c cois Bonnarel, Mark Kettenis , Mireille Louys}

\previousversion{}




\begin{document}

\begin{abstract}
This is a proposed extension to the ObsCore specification for data description, discovery and selection of radio data.
\end{abstract}

\section*{Acknowledgments}

The authors would like to thank all the participants in DM-WG and Radioastronomy-IG discussions
for their ideas, critical reviews, and contributions to this document.
We acknowledge also the support of  ESCAPE (European Science Cluster of Astronomy
and Particle Physics ESFRI Research Infrastructures) funded by the EU Horizon
2020 research and innovation program (Grant Agreement no 824064).
% are there other grants from partners to mention here?

\section{Introduction}


ObsCore specification \citep{2017ivoa.spec.0509L} defines both a minimal datamodel to describe datasets
and a table consistent with the model which can be served by TAP services. It has been successful
to define a lot of data discovery services in astronomy.

The emergence  of  the Radioastronomy Interest Group in the IVOA in April 2020 confirmed the strong
interest of the radio astronomy community to distribute their data in the VO. Many teams now
distribute their data using VO standards\footnote{https://ivoa.net/documents/Notes/RadioVOImp/index.html}.
While reduced radio data products, such as images or spectral cubes,%(and single dish data - {\it is that true ?} )
are mostly covered by the ObsCore model, the lower level observational data
(interferometric visibilities, single dish data in SDFITS, filterbank or whatever other specific formats) require additional description parameters.
For interferometry, this need  was already exposed
in 2010 by Anita Richards in a IVOA Note  "Radio interferometry data in the VO" 
\footnote{https://wiki.ivoa.net/internal/IVOA/SiaInterface/Anita-InterferometryVO.pdf} which captures precisely the requirements in the radio community.
 
 Various discovery use cases have been collected in the radio community and gathered in the \ref{ADQLusecases} appendix.
  The current specification suggests addition of new features in the ObsCore metadata profile to fill the gap.
% Mireille an attempt to widen the scope ??
With the expansion of large radio astronomy projects such as LOFAR, NenuFAR, the future SKA, ngVLA, etc...
and the emergence of interesting research topics matching data in all electromagnetic regimes, the
Virtual Observatory framework can facilitate a wider access to radio data for experts and
non-specialists radio astronomers in order to support collaborations in multi-wavelength,
multi-messenger astronomy.

%mireille:
% the needs for today as explored in recent meetings in Escape framework can also be summarized here
% add a short paragraph here
%Although the interferometry data are nowadays dominant in radio field, similar issues occur for
%optical interferometry data. The extension we are defining here is valid for both domains.



%\includegraphics[width=0.9\textwidth]{archdiag.png}
%\label{fig:architecture}
%\end{figure}

% Mireille : it seems this is not the good diagram but the one of SIA
% change architexture diag

%The role of this Visibility ObsCore extension in the IVOA is very close to ObsCore itself. It's only a fine grain improvement of this specification. 
% Mireille's  suggestion
This Radio ObsCore extension in the IVOA, relies  on ObsCore itself and expands it
to allow data providers to describe their radio data further the  ObsTAP framework.
Its goal is to clarify how  ObsCore metadata  can be used in the  radio context and to add new specific features to the existing ObsCore metadata.


\section{Radio data specifities from the Data Discovery point of view}
\label{sec:specificities}


On the lower end of the radio spectrum, radio astronomers generally make use of
frequencies for designating the spectral ranges of their observation. The standard
ObsCore attributes \emph{em\_min, em\_max} are expressed in wavelength and are not really convenient.
That's why we should also provide their translation into frequencies.

Receivers with a (ultra)wide bandwidth, up to tens of GHz, are
nowadays commonly used for both interferometric and Single Dish (herefater SD) radio observations.
Given that the spatial field of view and resolution linearly depend on wavelength, these quantities may significantly vary across the observed bandwidth in a radio observation.
Generally only a representative value (for instance the
%median
receiver nominal frequency) for these two parameters can be given. It is
noticeable that this is the case for any measuring system allowing a large interval of
$\lambda/D$ (where $\lambda$ and $D$ are the wavelength and the measuring system
aperture scale).
 
Similarly, the resolution power quantity, commonly provided to describe optical spectroscopic data, is generally not used in the radio domain.
Instead one could introduce a new ObsCore element for the absolute spectral resolution, in frequency unit, for which a representative value for each observation can be given.

Modern radio instrumentation offers the possibility of several  spectral windows within the same observation with significant separation or different resolutions.
Such observations may be represented at the highest granularity as  a set of combined data sets represented by several entries in an ObsCore Table. However it's up to data provider to decide which level of granularity is best adapted in order to optimize data discoverability and ease data access, depending on the scientific content of the observation.
%(see Sect. \ref{subsec:sd} for an example).



\subsection{Single dish data}\label{subsec:sd}

Single Dish observations can be done with different types of receiving systems. Typical frontends are mono-feed, multi-feed and phased array feed (PAF), the last two suitable to efficiently span wider parts of the sky.
Data can be acquired by various backend systems providing either the total intensity (integrated over the whole available bandwidth) or the spectroscopic/spectropolarimetric intensity (acquired in each spectral channel/sample).
Data are saved as raw counts resulting from the digitization of the voltage signal measured by the receiving system.
Commonly-used SD data formats are registered FITS standard conventions (FITS, SDFITS and MBFITS) or unregistered conventions like the standard FITS-based format delivered by the INAF radio telescopes.

The combination of telescope, frontend and backend permits the realization of various observing strategies characterized by specific spatial and/or spectral patterns.
Typical SD observing strategies are: on source, frequency switching, ON-OFF observations, raster or on-the-fly (OTF) mapping, raster or OTF cross-scan, skydip calibrations, see Fig~\ref{fig:SD}. For each spatial position in the observation, SD data gather emission for any of the spectral samples in the given frequency band and polarization.
If multi-feed/PAFs are used, a set of spatial positions are simultaneously measured. Scan modes should be described in ObsCore in order to allow astronomers to better understand the structure of the data which will be retrieved.

Spatial resolution in the SD case is intended as the beam size. This holds true for any type of receivers, since also for multi-feed/PAF ones the spatial resolution is ruled by the size of the individual beam.

Contrary to what usually happens for  interferometric observations, for some radio telescopes a SD observation (scan) contains only one scientific target (for example INAF ones). In any case, each target in an observation is listed as a separate entry in an ObsCore Table sharing the same \emph{obs\_id}.

Complex frequency setups are possible in the same observation, as already mentioned in Sect. \ref{sec:specificities}.

The ObsCore parameter t\_resolution, defined as the minimal interpretable interval between two points along the
time axis (being it an average or representative value), has a limited application for SD data except for on-source tracking
observations like those for pulsar/FRB studies.
Typically, time is not an independent variable in SD measurements and it can be saved together with spatial/spectral/intensity
information as a timestamp associated to each data sample
A more comprehensive discussion on ObsCore parameters for time-domain data is given in the Pulsar
and FRB Radio Data Discovery and Access IVOA Note\footnote{\url{https://wiki.ivoa.net/internal/IVOA/RadioastronomyInterestGroupFifthVirtualMeeting/PulsarRadioDataAccess.pdf}}.
%Even in the case of on-source tracking, time information in SD data is not intended for time domain studies.




\begin{figure}[H]
\centering

\includegraphics[width=0.9\textwidth]{SingleDish.png}
\caption{Single Dish Observation scan modes}
\label{fig:SD}
\end{figure}



\subsection{Visibility data }
\label{sec:visibility}

Visibility data are sets of complex numbers corresponding to the amplitude and phase
of correlation coefficients measured between pair of antennas (i.e., a baseline), at
a given time, a given wavelength or polarisation. The visibility data are a sparse
representation of the observed sky. The visibility data sets can be processed to obtain
interferometric images, through inverse Fourier algorithms. Each visibility measurement
corresponds to an interferometric fringe system on the sky.

The imaging algorithms include a calibration step allowing to set the center of the
reconstructed image, setting this direction as a phase reference. The visibilities
are then usually represented in a spatial frequency plane, called the \emph{uv} plane,
whose orientation is perpendicular to phase reference direction. The instantaneous PSF
(Point Spread Function) of an interferometer is the Fourier transform of all baselines
sampled in the \emph{uv} plane. Hence, the quality of the reconstructed images are
directly related to the set of baselines used for the measurements.

Visibility data are usually organised as sets of matrices for various phase references
(i.e., pointing, or fields) and configuration of the baselines, such as their
distances and orientations. Such matrices may or may not be regularly sampled in time,
wavelength and polarisation.

%How can the ObsCore parameters describe the characterization of these observations?
As for any other observation product described with ObsCore, the description may be split into
several records in the ObsCore table, when ObsCore parameters cannot represent the
variety of the observation results coverage (e.g., if there are several observed ``fields'',
requiring different s\_ra and s\_dec value, or various groups of spectral bands, etc.)

We consider that consistent ObsCore records as described above defines datasets with
a dataproduct type set to ``visibility''.


Contrary to what occurs with direct imaging observations, the PSF of the interferometer
is filtering spatial scales (large scales, when the small baselines are insufficiently
sampled; and vice versa for small scales with long baselines).
For large spectral ranges, the variations of the field of view and the spatial resolution
along the axis may become so large that the typical value cannot be sufficient to
characterize the dataset. Ranges of values for such parameters are required to accurately
describe such datasets.

The quality of the data strongly depends from the distribution of the visibility measurements
in the \emph{uv} plane : the more complete the \emph{uv} sampling plane, the better the reconstructed image.
The \emph{uv} plane distribution can be characterized by several numbers.
The minimal and maximum distance between measurements in the \emph{uv} plane provide assessments for
spatial resolution and largest angular scale.
Beside this a \emph{uv} plane filling factor of the distribution will allow to predict the quality
of reconstruction of the image in the distance plane (sky).
Eventually, the ellipticity of the distribution is a measure of the distortions that can
affect the reconstruction.

Radio astronomers also check the quality of the visibility data by looking at some maps of
the data structure. The \emph{uv} coverage map can show how complete and regular is the sampling in
the \emph{uv} plane and give an hint of resolution and maximum angular scale.
The visualisation of the dirty beam, which is the Fourier transform of the \emph{uv} sampling
function gives an hint of the intrinsic quality of possible reconstruction. As maps they are
not queriable. So links to these kind of maps will not be exposed in the extension
table but only via a DataLink service.

If none of these \emph{uv} characterization features are available to be exposed in the service
we can still predict ranges of some of those by using some kind of facility description. 
Important features are the antenna diameter (or maximum antenna diameter), the number of
antennas and the minimum and maximum distance between antennas of the array.

In addition to these specifities most of the scan modes shown on figure~\ref{fig:SD} also
apply to some interferometry observations and should be described.

\section{ObsCore attributes definition valid for radio data}
\label{sec:ObsCoreRadDef}

%Some mandatory or optional parameters will have peculiar estimation for visibility data.
For radio data some of the definitions on ObsCore data model attributes need to be adjusted
to fit the peculiarity of metadata for datasets partition, \emph{uv} space, etc.

\subsection{obs\_id}

Astronomers usually know what they identify as a single observation: a complex set of
measurements made in a given sequence of time. obs\_id should define unambiguously each
observation.

\subsection{obs\_publisher\_did}

Radio data observations can be split in several subparts with homogeneous spatial,
time, spectral coverage intervals, spectral resolution, etc. Each part can be described by
a single ObsCore dataset and has its own obs\_publisher\_did. It has to be unique in the
Virtual Observatory domain.

\subsection{s\_fov}
\label{sec:fov}

A typical value for the field of view size is to be computed on the observation by taking into account the sky scan geometry and receiver type in use.
s\_fov coincides with the instantaneous field of view $\lambda / D$ only for pointed observations (for instance, an ON in the SD case) obtained with a mono-feed receiver. In this case, $\lambda$ is the
%mid value of the spectral range
receiver nominal wavelength and D coincides with the telescope diameter (SD case) or the largest diameter of the array antennae or telescopes (interferometric case).



\subsection{s\_resolution}
\label{sec:res}
In the case of SD using mono- or multi-feed/PAF receivers this is the beam size inferred from the wavelength and telescope diameter.
In the case of interferometry, a typical value for the spatial resolution will be given by $\lambda / L$ where $\lambda$
is the %mid value of the spectral range
receiver nominal wavelength and L is the longest distance in the \emph{uv} plane.

For beamforming applied to SD \emph{s\_resolution} is set by the size of one individual electronically-formed beam, while in the interferometric
case it is ruled by the maximum distance among the stations.
 
 
\subsection{s\_region}
For single dish data it will strongly depend on the scanning mode and type of receiver in use.
This shape will be the typical contour of the detectable beam for interferometry. Of course it cannot be accurate.

\subsection{o\_ucd}

In the current UCD vocabulary (UCD1+ controlled vocabulary - Updated List of Terms Version 1.5) there appear to be no primary words suitable to describe raw SD data. o\_ucd=phot.flux.density, which fits well for appropriate data, does not seem appropriate, since the single dish measured quantity is expressed in raw counts coming from the digitisation of a voltage signal generated in the receiver chain by the incoming electromagnetic field. Further discussion on o\_ucd is ongoing within the Semantics WG.

In the case of visibility data the "observable" is a complex number representing Fourier
coefficients of the image Fourier transform. Its UCD string is \emph{stat.fourier}.

\subsection{t\_exptime}
Total duration of the observation for the given dataset or ObsCore entry. Example: in case of multiple targets, \emph{t\_exptime}
will be computed for each source and written in the appropriate ObsCore Table entry.



\subsection{t\_resolution}
%Not applicable for single dish data (see Sect. \ref{subsec:sd}).
The ObsCore parameter t\_resolution (see Sect. \ref{subsec:sd}) has a limited application for SD data
except for on-source tracking observations like those for pulsar/FRB studies and could be set to the
exposure time or could be NULL. For time-domain data, \emph{t\_resolution} can be set according to the Pulsar
and FRB Radio Data Discovery and Access IVOA Note \footnote{\url{https://wiki.ivoa.net/internal/IVOA/RadioastronomyInterestGroupFifthVirtualMeeting/PulsarRadioDataAccess.pdf}}.

For interferometric observations it is the integration time set at the correlation level.


\subsection{dataproduct\_type and dataproduct\_subtype}

Radio astronomy data cover a wide variety of dataproduct types from visibility for raw interferometry data to cubes, images, spectra, time series
or even measurements (in the case of single dish on scan mode). Single dish observations in some modes show specifities which are not covered by
the current ObsCore \emph{dataproduct\_type} vocabulary. This is the case of spatial profiles obtained with on the fly cross scan or of the
tables of flux measurements obtained on a regular spatial grid but with specific time stamp for each spot as in the raster map  mode.
A new external standard IVOA vocabulary is currently defined for dataproduct types \footnote{https://www.ivoa.net/rdf/product-type/2023-06-26/product-type.html}
and tackles some of these specificities. However some of them SHOULD be covered in the \emph{dataproduct\_subtype} attribute if no new term is introduced in the standard vocabulary.

\section{ObsCore extension specific for radio data}

Table \ref{tab:ExtensionAtt} shows the %additional
querying parameters we propose to add to ObsCore in order to better describe radio single dish and visibility data.
The last column indicates if the attribute is useful for all radio datasets or only for visibilities, beam forming, or single dish data.

\subsection{spatial parameters}
\emph{s\_fov\_min, s\_fov\_max} are estimated like the typical value (see subsection \ref{sec:fov}). 
In the case of SD pointed observations with mono-feed receivers and the majority of interferometric observations the minimum and maximum
$\lambda$ values in the spectral range(s) will be used in the formula. In the case of mapping scans or multi-feed/PAF receivers
\emph{ s\_fov\_min} and \emph{s\_fov\_max} are derived as the minimum and maximum sizes of the circular region encompassing the covered area.


\emph{s\_resolution\_min, s\_resolution\_max} are estimated like the typical value (see subsection \ref{sec:res}) where $\lambda$ is replaced by the minimum and maximum wavelength of the spectral range(s). The size D is the telescope diameter for SD observations and the largest distance in the \emph{uv} plane. Beam forming may represent an exception to this rule, see \ref{sec:res}.


In the case of interferometry, the \emph{s\_maximum\_angular\_scale} is estimated as $\lambda/l$ where $\lambda$ is the typical
wavelength and l is the typical smallest distance in the \emph{uv} plane.


\subsection{frequency characterization}

As was stated above (\ref{sec:specificities}) radio astronomers use frequency quantities to characterize their datasets. Therefore we introduce three additional parameters in the extension : \emph{f\_min} and \emph{f\_max} for spectral ranges and \emph{f\_resolution} for absolute spectral resolution .

\textit{To avoid inconsistency between the core attributes \emph{em\_min, em\_max} and \emph{em\_respower} and these additional quantities we suggest to compute these new quantities when building a view on top of the basic ObsCore table. The frequency attributes MUST be expressed in Khz to allow interoperability.} 


\subsection{Spatial frequency coverage for visibilities  }
These parameters are valid for interferometry only.

%uv\_distance\_min and uv\_distance\_max are straigthforward.
The absolute \emph{uv}\_distance\_min and \emph{uv}\_distance\_max  in the \emph{uv} plane give some outlier minimum and maximum scale in some direction.
%are evaluated by fitting an ellipse on the visibilities present in the uv plane.

% Mireille but still for the spec it is good to redefine them here
To compute the ellipse's eccentricity of the \emph{uv} distribution a principal component analysis
(PCA) with 2 components is performed over the data points sampling the \emph{uv} plane to select the
main axis of data scattering. 
The first component is used to rotate the distribution of uv in a way that the major variation
of the distribution is leaning towards the $x$ axis of a bi dimensional $xy$ Cartesian plane.
The major axis length and the minor axis length of the ellipse are therefore defined as the
semi distance between the most positive point along the $x$/$y$ axis and the most negative point
among the $y$ axis. For instance, if the range of the rotated UV will cover on the $x \in [-10,
10]$ the major axis distance would be 10, a similar procedure is done on the y axis.

This procedure allows the definition of the \emph{uv} distribution eccentricity:

uv\_distribution\_exc computed as follows:
\begin{equation}
uv\_distribution\_exc = \sqrt{1-\frac{b^2}{a^2}}
\end{equation}
where a is the major axis length and b is the minor axis length.
The filling factor of the \emph{uv} plane (hereafter uv\_distribution\_fill) is computed as the average
number of samples found in a $N^{uv}_{samples}$x$N^{uv}_{samples}$ equi-spaced grid enclosing the
rotated ellipse. In formulas, the boundaries of a cell (i,j) are defined by the boundaries
\begin{equation}
u \in [u_{min} + \frac{u_{max} - u_{min}}{N^{uv}_{samples}} \cdot i , u_{min} + \frac{u_{max} -
u_{min}}{N^{uv}_{samples}} \cdot (i + 1)]
\end{equation}
and
\begin{equation}
v \in [v_{min} + \frac{v_{max} - v_{min}}{N^{uv}_{samples}} \cdot j , v_{min} + \frac{v_{max} -
v_{min}}{N^{uv}_{samples}} \cdot (j + 1)]
\end{equation}
where $u_{max}$/$v_{max}$ are the respective maximum u/v of the \emph{uv} sample and
$u_{min}$/$v_{min}$ is the minimum u/v of the \emph{uv} sample.

Given the above boundaries the number of samples within a cell (i,j) will be $n^{uv}_{i,j}$
and uv\_distribution\_fill will be then computed as
\begin{equation}
uv\_distribution\_fill = \frac{\sum^{N^{uv}_{samples}}_{i=1} \sum^{N^{uv}_{samples}}_{j=1}
n^{uv}_{i,j} }{(N^{uv}_{samples}) ^ 2},
\end{equation}

in the preliminary analysis $N^{uv}_{samples} = 1000$.


% Mireille: moved time param below uv space because uv plane deals with spatial features

\subsection{time parameters}
\emph{t\_exp\_min, t\_exp\_max }and \emph{t\_exp\_mean} are added in the case of variation in the individual timestamps
duration. This is usually not the case for SD data and these parameters will be set to NULL.


\subsection{observation and instrumental parameters}
These parameters are intended to describe the main telescope(s) characteristics for both SD antennas and interferometers. 
Such instrumental characteristics give an approximate idea on the spanned angular scales, field of view, product types, etc.

The more global parameter to define is the instrument type allowing to discriminate single dish observations from interferometry or beam forming.

Parameters \emph{instrument\_ant\_number, instrument\_ant\_min\_dist} and  \emph{instrument\_ant\_max\_dist} are related to interferometers only while 
\emph{instrument\_ant\_diameter, instrument\_feed } are valid also for SD.
We note that \emph{instrument\_feed} could also  account for the number of beams in the case of a beam forming/PAF receiver.


The scanning strategy adopted in an observation is described by the parameter \emph{scan\_mode}. This parameter covers both spatial
and frequency scanning modes (see Sect.~\ref{subsec:sd} for details).
It is applicable to  SD as well as  interferometry cases.

Pointing mode distinguishes targeted observations from  fixed alt-azimuth or wobble ones. The ObsLocTAP specification \citep{2021ivoa.spec.0724S} defines the term \emph{tracking\_type} for describing this as well as a  vocabulary for these modes. 
We include the same term here in the extension.


\subsection{Auxiliary datasets useful for data quality estimation}

Auxiliary datasets such as  \emph{uv} distribution map, dirty beam maps, frequency/amplitude plots, phase/amplitude plots are useful for astronomers to check data quality.
In that case DataLink \citep{2015ivoa.spec.0617D} may provide a solution to attach these auxiliary data to ObsCore records. The semantics FIELD in the \{link\}
response  will contain \#auxiliary  for links to these maps or plots while  the content\_qualifier FIELD introduced from 1.1  could contain a vocabulary as defined in ( TO BE COMPLETED ).


\section{The ivoa.obscore\_radio table}
\label{sec:implementation}
% commented below: Markus version
%The ObsCore Extension for Radio is accessed as a table within a TAP
%\citep{2019ivoa.spec.0927D} service.\todo{I think we should require a
%few UDFs on TAP services: ivo\_interval\_overlaps, ivo\_specconv}  At this
%point, the name of this table is fixed to \verb|ivoa.obs_radio|.
%However, in order to allow for later evolution -- which may allow
%multiple compliant tables per service --, this name should not be used
%for discovery (see sect.~\ref{sec:registry}).
%
%A TAP service that has \verb|ivoa.obs_radio| must also have a table
%compliant to any version 1 of ObsCore, i.e., a table
%\verb|ivoa.obscore|; the two tables must share exactly the columns such
%that a \verb|NATURAL JOIN| will yield per-dataset rows of obscore and
%radio extension metadata.  Which columns these are is site-specific.
%
%To ensure that all compliant services can execute the same queries,
%all columns in table~\ref{tab:ExtensionAtt} must be present in such a
%table, although any may be NULL.  Additional columns may be present, and
%at least a foreign key into \verb|ivoa.obscore| will typcially
%user-visible\todo{can we make rules such that such additional columns
%will not interfere with later extensions?}.
%
%The intention is that clients will write queries like
%\begin{lstlisting}
%SELECT [any obscore and obs_radio columns or expressions]
%FROM
%  ivoa.obscore
%  NATURAL JOIN ivoa.obs_radio
%WHERE
%  [constraints]
%\end{lstlisting}
The ObsCore Extension for Radio data is accessed in a table containing also ObsCore fields within a TAP
\citep{2019ivoa.spec.0927D} service.\todo{I think we should require a
few UDFs on TAP services: ivo\_interval\_overlaps, ivo\_specconv} At this
point, the name of this table is fixed to \verb|ivoa.obscore_radio|.
However, in order to allow for later evolution -- which may allow
multiple compliant tables per service --, this name should not be used
for discovery (see sect.~\ref{sec:registry}).

A TAP service that has \verb|ivoa.obscore_radio| must also have a table
compliant to any version 1 of ObsCore, i.e., a table
\verb|ivoa.obscore|; containing only the basic ObsCore attributes.
This table can be implemented as a simple view on the \verb|ivoa.obscore_radio|
table

To ensure that all compliant services can execute the same queries,
all columns in table~\ref{tab:ExtensionAtt} must be present in such a
table, although any may be NULL. Additional columns may be present\todo{can we make rules such that such additional columns
will not interfere with later extensions?}.

The intention is that clients will write queries like
\begin{lstlisting}
SELECT [any obscore and obs_radio columns or expressions]
FROM
ivoa.obscore_radio

WHERE
[constraints]
\end{lstlisting}
% commented : Markus initial PR
%without having to know the exact foreign key(s) in use on any given
%service.  This means that while one service can opt to join on
%\verb|obs_publisher_did| (we expect this to be a very common choice),
%other services may choose to join on, say, artificial keys.

% main text previous Markus PR
%The ObsCore extension for radio (including or not visibility data) described above SHOULD  be added to the main ObsCore table.
%The standardID for this extented table will be ivo://ivoa.net/std/ObsCore\#radioExt-1.0". This implies that the  core ObsCore attributes follow version 1.1 of the Obscore standard.
%\textit{ In practice a  table containing only the extension attributes  MAY be added to the same schema.
%The full extension table will then be a view joining the core table and the  table with extension attribue only via an extended ObsTAP ADQL query.
%A single dataset in each observation will be associated to a single row in ObsCore. It will be identified by a unique obs\_publisher\_did.
%This obs\_publisher\_did can be used as a foreign key to join the main table and the extension table.}
%
%In the registry, the Obscore table and the Extended table MUST be described in the tableset of the service.
%They will show respectively the ObsCore Model utype "ivo://ivoa.net/std/ObsCore\#core-1.0" an the extended Obscore Model utype "ivo://ivoa.net/std/ObsCore\#radioExt-1.0" . 


%first table 
        
\begin{landscape}
\begin{longtable}{l  p{4cm} p{4cm} p{4.5cm} l l l}
\sptablerule
\textbf{column name}&\textbf{definition}&\textbf{utype}&\textbf{ucd}&\textbf{unit}&\textbf{validity}\cr
\sptablerule
\sptablerule
\texttt{ s\_resolution\_min}&\texttt{ Angular resolution, longest baseline and  max frequency dependant}&{ Char.SpatialAxis.\newline Resolution.Bounds.\newline Limits.LoLim}&{pos.angResolution;stat.min}&{arcsec}&radio\cr
\sptablerule
\texttt{s\_resolution\_max}&\texttt{Angular resolution, longest baseline and min frequency dependant}&\texttt{Char.SpatialAxis.\newline Resolution.Bounds.\newline Limits.HiLim}&{pos.angResolution;stat.max}&arcsec&radio\cr
\sptablerule
\texttt{s\_fov\_min}&\texttt{field of view diameter,  min value, max frequency dependant}&\texttt{Char.SpatialAxis.\newline Coverage.Bounds.\newline Extent.LowLim}&{phys.angSize;instr.fov;\newline stat.min}&deg&radio\cr
\sptablerule
\texttt{s\_fov\_max}&\texttt{field of view diameter,  max value, min frequency dependant}&\texttt{Char.SpatialAxis.\newline Coverage.Bounds.\newline Extent.HiLim}&{phys.angSize;instr.fov;\newline stat.max}&deg&radio\cr
\sptablerule
\texttt{s\_maximum\_angular\_scale}&\texttt{maximum scale in dataset, shortest baseline and  frequency dependant}&\texttt{Char.SpatialAxis.\newline Resolution.Scale.\newline Limits.HiLim}&{phys.angSize;stat.max}&arcsec&interferometry\cr
\sptablerule
\texttt{f\_min}&\texttt{spectral coverage min in frequency}&\texttt{Char.SpectralAxis.\newline Coverage.Bounds\newline Limits.LoLim}&{em.freq;stat.min}&Mhz&radio\cr
\sptablerule
\texttt{f\_max}&\texttt{spectral coverage max in frequency}&\texttt{Char.SpectralAxis.\newline Coverage.Bounds\newline Limits.HiLim}&{em.freq;stat.max}&Mhz&radio\cr
\sptablerule
\texttt{f\_resolution}&\texttt{absolute spectral resolution in frequency}&\texttt{Char.SpectralAxis.\newline Coverage.Bounds\newline Limits.HiLim}&{em.freq;stat.max}&Khz&radio\cr
\sptablerule
\texttt{t\_exp\_min}&\texttt{minimum integration time per sample}&\texttt{Char.TimeAxis.\newline Sampling.Extent\newline LoLim}&{time.duration;obs.exposure;\newline stat.min}&s&radio\cr
\sptablerule
\texttt{t\_exp\_max}&\texttt{maximum integration time per sample}&\texttt{Char.TimeAxis.\newline Sampling.Extent\newline HiLim}&{time.duration;obs.exposure;\newline stat.max}&s&radio\cr
\sptablerule
\texttt{t\_exp\_mean}&\texttt{average integration time per sample}&\texttt{Char.TimeAxis.\newline Sampling.Extent\newline HiLim}&{time.duration;obs.exposure\newline stat.mean}&s&radio\cr
\sptablerule
\texttt{uv\_distance\_min}&\texttt{minimal distance in uv plane}&\texttt{Char.UVAxis.\newline  Coverage.Bounds.\newline Limits.LoLim}&stat.fourier;pos;stat.min&m&interferometry \cr
\sptablerule
\texttt{uv\_distance\_max}&\texttt{maximal distance in uv plane}&\texttt{Char.UVAxis.\newline  Coverage.Bounds.\newline Limits.LoLim}&stat.fourier;pos;stat.max&m&interferometry \cr
\sptablerule
\texttt{uv\_distribution\_ecc}&\texttt{eccentricity of uv distribution}&\texttt{Char.UVAxis.\newline  Coverage.Bounds.\newline Eccentricity}&stat.fourier;pos&&interferometry \cr
\sptablerule
\texttt{uv\_distribution\_fill}&\texttt{filling factor of uv distribution}&\texttt{Char.UVAxis.\newline  Coverage.Bounds.\newline FillingFactor}&stat.fourier;pos;arith.ratio&&interferometry \cr
\sptablerule
\texttt{uv\_distribution\_map}&\texttt{uv distribution map}&\texttt{Char.UVAxis.\newline  Sampling.\newline Sensitivity.Map}&stat.fourier;pos;meta.ref.url&&interferometry \cr
\sptablerule
%\texttt{s\_beam\_dirty}&\texttt{dirty beam}&\texttt{- (via DataLink}&{via DataLink}&&interferometry\cr
%\sptablerule
%\texttt{phase\_amplitude}&\texttt{phase amplitude plot}&\texttt{via DataLink}&{via DataLink}&&interferometry\cr
%\sptablerule
%\texttt{frequency\_amplitude}&\texttt{dirty beam}&\texttt{via DataLink}&{via DataLink}&&interferometry\cr
%\sptablerule
\caption{ObsCore extension proposal for radio data}
\label{tab:ExtensionAtt}
\end{longtable}
\end{landscape}

%second table
\begin{landscape}
\begin{longtable}{l  p{4cm} p{4cm} p{4.5cm} l l l}
\sptablerule
\textbf{column name}&\textbf{definition}&\textbf{utype}&\textbf{ucd}&\textbf{unit}&\textbf{validity}\cr
\sptablerule
% mireille here for antennas features , it is clear it belongs to instrument.
%Not useful to use the term in the name .
%\texttt{instrument\_ant\_number}&\texttt{number of antennas in array}&\texttt{Provenance.ObsConfig.\newline Instrument.Array.\newline AntNumber}&instr.baseline;meta.number& \cr
\sptablerule
\texttt{instrument\_ant\_number}&\texttt{number of antennas in array}&\texttt{Provenance.ObsConfig.\newline Instrument.Array.\newline AntNumber}&meta.number;instr.param&&interferometry, beamforming \cr
\sptablerule
% same for all antenae features
\texttt{instrument\_ant\_min\_dist}&\texttt{minimum distance between antennas in array}&\texttt{Provenance.ObsConfig.\newline Instrument.Array.\newline MinDist}&instr.baseline;stat.min&m&interferometry \cr
\sptablerule
\texttt{instrument\_ant\_max\_dist}&\texttt{maximum distance between antennas in array}&\texttt{Provenance.ObsConfig.\newline Instrument.Array.\newline MaxDist}&instr.baseline;stat.max&m&interferometry \cr
\sptablerule
\texttt{instrument\_ant\_diameter}&\texttt{diameter of telecope or antennas in array}&\texttt{Provenance.ObsConfig.\newline Instrument.Array.\newline Diameter}&instr.param&m&radio \cr
\sptablerule
\texttt{instrument\_feed}&\texttt{number of feeds}&\texttt{Provenance.ObsConfig.\newline Instrument.Feed}&instr.param&& radio  \cr
\sptablerule
\texttt{scan\_mode}&\texttt{scan mode (on-off, \newline raster map, on-the-fly map,...)\newline }&\texttt{Provenance.\newline Observation.\newline sky\_scan\_mode}&instr.param&& radio \cr
\sptablerule
\texttt{tracking\_mode}&\texttt{targeted, alt-azimuth, wobble, ...)\newline }&\texttt{Provenance.\newline Observation.\newline tracking\_mode}&instr.param&& radio \cr
\caption{ObsCore extension proposal for instrumental parameters for radio data}
\label{tab:ExtensionAtt_instrumental}
\end{longtable}
\end{landscape}
% c!ommented below : Markus PR version
%\section{Registry Aspects}
%\label{sec:registry}
%
%Services compliant with this specification are registered using
%VODataService \citep{2021ivoa.spec.1102D} tablesets.  Compliant tables
%use the utype
%$$
%\hbox{\nolinkurl{ivo://ivoa.net/std/obsradio#table-1.0}.}
%$$
%
%While it is admitted that the table only sits in the tableset of the
%embedding TAP service, implementors are urged to use a seperate registry
%record with the main TAP service as an auxiliary capability
%\citep{2019ivoa.spec.0520D}.  In this way, meaningful information
%on coverage in space, time, and spectraum as per VODataService 1.2 can
%be communicated to the Registry, which, again, data providers are urged
%to do.  There is no expectation that the coverage information only
%pertains to data with entries in \verb|ivoa.obs_radio|, i.e., it may be
%a copy of the coverage of the full Obscore table.\todo{Is that
%acceptable?  Or should we require pure radio coverage?}
%
%A sample registry record for an obs\_radio table comes with this
%specification\footnote{\auxiliaryurl{sample-record.xml}}.
%
%To obtain access URLs of all TAP services that have compliant tables
%together with their table names (which in this major version are fixed
%to \verb|ivoa.obs_radio|), use a RegTAP \citep{2019ivoa.spec.1011D}
%query like:
%
%\begin{lstlisting}
%SELECT DISTINCT(access_url), table_name
%FROM rr.res_table
%  NATURAL JOIN rr.capability
%  NATURAL JOIN rr.interface
%WHERE
%  standard_id LIKE 'ivo://ivoa.net/std/tap%'
%  AND intf_role='std'
%  AND table_utype LIKE 'ivo://ivoa.net/std/obsradio#table-1.%'
%\end{lstlisting}

\section{Registry Aspects}
\label{sec:registry}

Services compliant with this specification are registered using
VODataService \citep{2021ivoa.spec.1102D} tablesets. Compliant tables
use the utype
$$
\hbox{\nolinkurl{ivo://ivoa.net/std/ObsCore#radioExt-1.0}.}
$$

While it is admitted that the table only sits in the tableset of the
embedding TAP service, implementors are urged to use a seperate registry
record with the main TAP service as an auxiliary capability
\citep{2019ivoa.spec.0520D}. In this way, meaningful information
on coverage in space, time, and spectraum as per VODataService 1.2 can
be communicated to the Registry, which, again, data providers are urged
to do. There is no expectation that the coverage information only
pertains to data with entries in \verb|ivoa.obscore_radio|, i.e., it may be
a copy of the coverage of the basic Obscore table.\todo{Is that
acceptable? Or should we require pure radio coverage?}

A sample registry record for an obscore\_radio table comes with this
specification\footnote{\auxiliaryurl{sample-record.xml}}.

To obtain access URLs of all TAP services that have compliant tables
together with their table names (which in this major version are fixed
to \verb|ivoa.obscore_radio|), use a RegTAP \citep{2019ivoa.spec.1011D}
query like:

\begin{lstlisting}
SELECT DISTINCT(access_url), table_name
FROM rr.res_table
NATURAL JOIN rr.capability
NATURAL JOIN rr.interface
WHERE
standard_id LIKE 'ivo://ivoa.net/std/tap%'
AND intf_role='std'
AND table_utype LIKE 'ivo://ivoa.net/std/ObsCore#radioExt-%'
\end{lstlisting}

\appendix

\section{Science use cases and translation to ADQL queries  based on ObsCore Radio extension  }
\label{ADQLusecases}


\textit{ The  queries in subsections \ref{sec:s_resolution_max} and \ref{sec:s_fov_min}
can be run with s\_resolution\_min and s\_fov\_max instead of s\_resolution\_max and 
 s\_fov\_min respectively, to obtain results in which at least part of the dataset satisfy the condition.\\
For instance, consider datasets with a large observed bandwidth, like those taken with wide/ultrawide band receivers.
The condition s\_resolution\_max<d\_typical may not be satisfied. At the same time, the user may want to select those 
dataset where the desired condition on spatial resolution is matched at least partially  within the observed band.}

\textit{Sections from \ref{sec:instrument-mode} illustrate instrument and observing modes parameters. They must  be completed by science cases.}

%\subsection{Science case }
\subsection{Use case - s\_resolution\_min}
\label{sec:s_resolution_min}
\textit{Comparative study of low surface brightness extended emission surrounding compact sources, like for instance Virgo A.
In order to avoid washing out of diffuse emission the spatial resolution should not be too high.
Select any dataset map (raster or on-the-fly) with s\_resolution\_min larger than the desired spatial resolution of 1 arcmin.}
Show me all datasets satisfying: \\
I. Minimum spatial resolution > 0.017 deg \\
II Target Virgo A or position inside 15 arcmin from 187.7059308,+12.3911232 \\
III Scan mode is raster map or on-the-fly map

\begin{verbatim}
SELECT * FROM ivoa.obscore
NATURAL JOIN  ivoa.obscore-radio 
WHERE
s_resolution_min > 0.017 
 AND (target_name = 'Virgo A' OR
CONTAINS(POINT(s_ra, s_dec),CIRCLE,187.7059308,+12.3911232,0.25)) = 1)) 
AND  (scan_mode = 'raster map' OR scan_mode = 'on-the-fly map')
\end{verbatim}

\subsection{Use case - s\_resolution\_max}
\label{sec:s_resolution_max}
\textit{Morphological study of SNR IC443. Select any dataset obtained in map scanning mode (raster or on-the-fly) with s\_resolution\_max less than the desired spatial resolution of 1 arcmin.}

Show me all datasets satisfying: \\
I. Maximum spatial resolution < 0.017 deg \\
II Target IC443 or position inside 15 arcmin from 94.2500000,+22.5699997 \\

\begin{verbatim}
SELECT * FROM ivoa.obscore
NATURAL JOIN  ivoa.obscore-radio 
WHERE s_resolution_max < 0.017 
AND
 (target_name = 'IC443' OR
CONTAINS(POINT(s_ra,s_dec),CIRCLE(94.2500000,+22.5699997,0.25)) = 1)) 
AND (scan_mode = 'raster map' OR scan_mode = 'on-the-fly map')
\end{verbatim}

\subsection{Use case - s\_fov\_min - large field of views}
\label{sec:s_fov_min}
\textit{Select any dataset obtained in map scanning mode (raster or on-the-fly) with minimum field of view larger than 0.8 degree. For instance, investigate the region surrounding cluster Abell 194.}

Show me all datasets satisfying:\\
I. Minimum FOV > 0.8 deg \\
II Target name = Abell 194 or position inside 15 arcmin from 21.5054167, -1.3672221 \\
\begin{verbatim}
SELECT * FROM ivoa.obscore
NATURAL JOIN  ivoa.obscore-radio 
WHERE s_fov_min > 0.8 AND
(target_name = 'Abell 194' OR
CONTAINS(POINT(s_ra,s_dec),CIRCLE(21.5054167,-1.3672221,0.25)) = 1)) 
AND (scan_mode = 'raster map' OR scan_mode = 'on-the-fly map')
\end{verbatim}

\subsection{Use case - s\_fov\_min - extended target completly included}
\label{sec:s_fov_max}
\textit{Pictor A has a typical angular extension  of 8 arcminutes , select any image/cube completely containing the target}
Show me all datasets satisfying: \\
I. Target name = Pictor A \\
II. The circle defined by the minimum FOV of the dataset fully contains the circle delimiting Pictor A. \\
\begin{verbatim}
SELECT * FROM ivoa.obscore
NATURAL JOIN  ivoa.obscore-radio 
WHERE target_name = 'Pictor A' AND
CONTAINS(CIRCLE(79.9571789, -45.7788479,(8/60)/2),
CIRCLE(s_ra, s_dec, s_fov_min/2)) = 1)
\end{verbatim}

\subsection{Use case - dataproduct\_type}
\textit{Select all observations of calibrator source 3C48 with the Medicina Grueff radio telescope at frequencies in the range 20-21 GHz, to investigate its flux variability (light curve) in the years 2000-2023}
Show me all datasets satisfying: \\
I. Target name = 3C 48 \\
II. obs\_collection = ?INAF-Medicina, single dish?\\
III. Observed frequency in the range 20-21 GHz \\
IV. dataproduct\_type = spatial\_profile or scan\_mode contains map\\
V. 51544 < observation time (MJD) < 60309 \\


\begin{verbatim}
SELECT * FROM ivoa.obscore
NATURAL JOIN  ivoa.obscore-radio
WHERE obs_collection EQ 'INAF-Medicina, single dish' 
AND (em_min >=20000 AND em_max <= 21000) 
AND (dataproduct_type EQ 'spatial profile' OR scan_mode LIKE "*map*") 
AND  target_name = '3C48' 
AND  (t_min >= 51544 AND t_max <= 60309)
\end{verbatim}

\subsection{Use case - f\_resolution, frequency ranges}
\label{sec:FreqRanges}

\textit{Select any cube with spectral resolution better than 1 MHz and spectral range inside the 1 to 1.5 GHz band (around HI 21 cm line).}

Show me all datasets satisfying:\\
I. Product type is cube \\
II. spectral resolution better than 1 MHz \\
III. em\_min and em\_max such than frequency bounds are included in the 1GHz to 1.5GHz range \\
\begin{verbatim}
SELECT *, 299792458 / em_max as f_min,  299792458 / em_min as f_max 
FROM ivoa.obscore 
NATURAL JOIN  ivoa.obscore-radio
WHERE dataproduct_type = 'cube' AND
f_resolution < 1000000 AND
299792458 / em_max  > 1.0 e+9 AND
299792458 / em_min < 1.5e+9
\end{verbatim}

\subsection{Use case - frequency selection  for images }
\textit{Any image or cube observed at frequencies below 1 GHz.}\\ \\
Show me all datasets satisfying:\\
I. Product type is image or cube \\
II. higher spectral limit < 1Ghz
\begin{verbatim}
      SELECT * FROM ivoa.obscore
      WHERE (dataproduct_type = 'image' 
      OR  dataproduct_type = 'cube') 
      AND 299792458 / em_min  < 1e9
\end{verbatim}
% Mireille Louys: shall we also look for spectra with these frequency constrains ? 


 
% end section Alexandra's science case

% start of Mark's (VLBI) science cases

% These use NATURAL JOIN
% Should these restrict the dataproduct_type?
\subsection{Use case - high resolution data around FRB targets }

\textit{Give me high-resolution data on possible persistent radio sources within an arc second of FRB 121102:}

\begin{verbatim}
SELECT * FROM ivoa.obscore NATURAL JOIN ivoa.obscore-radio
WHERE CONTAINS(POINT(s_ra,s_dec),CIRCLE(82.99458,33.14794,0.0003)) = 1 
AND s_resolution_max < 0.001
\end{verbatim}

\subsection{Use case - reasonable fidelity}

\textit{Give me data on extended HI emission around the source 3C84 that can
be imaged with reasonable fidelity:}

\begin{verbatim}
SELECT * FROM ivoa.obscore 
NATURAL JOIN ivoa.obscore-radio
WHERE CONTAINS(POINT(s_ra,s_dec),CIRCLE(24.4212500,33.1588889,0.003)) = 1 
AND s_largest_angular_scale > 0.018 
AND uv_distribution_fill > 0.2 
AND uv_distribution_ecc < 0.75 
AND em_min < 0.21 
AND em_max > 0.21
\end{verbatim}

% end of Mark's (VLBI) science cases

\subsection{ use case - visibility data product and target object selection }
\textit{Any raw interferometry observation containing target Fornax cluster.}\\
\\
Show me all  observation identifiers satisfying:\\
I. DataType = visibility \\
II. Target Name = Fornax Cluster \\
\begin{verbatim}
      SELECT obs_id FROM ivoa.obscore
      WHERE dataproduct_type = 'visibility' 
      AND target_name = 'Fornax cluster'
\end{verbatim}



\subsection{Use case - maximum angular scale selection}
\textit{Any visibility dataset Within an arcec around FRB 121102  where  maximum angular scale is larger than 1 minute .}\\ \\
Show me all datasets satisfying:\\
I. Product type is visibility \\
II. Maximum angular scale  >  0.018 deg \\
III. Position within 1 arcsec of 82.99458 + 33.14794
\begin{verbatim}
      SELECT * FROM ivoa.obscore 
      NATURAL JOIN  ivoa.obscore-radio
      WHERE dataproduct_type = 'visibility' 
      AND s_largest_angular_scale_max  > 0.018 
      AND CONTAINS(POINT(s_ra,s_dec),CIRCLE(82.99458, 33.14794,0.0003)) = 1))
\end{verbatim}

\subsection{Use case - maximum angular scale selection with reasonable fidelity}
\textit{Any visibility data within an arcsec around FRB 121102 where  maximum angular scale is larger than 1 minute  and 
of good quality from the uv distribution point of view.}\\ \\
Show me all datasets satisfying:\\
I. Product type = visibility \\
II. Maximum angular scale  >  0.018 deg \\
III. uv distribution filling factor  > 0.7 \\
IV. uv distribtion factor eccentricity  < 0.1 \\
V. Position within 1 arcsec of 82.99458 + 33.14794
\begin{verbatim}
      SELECT * FROM ivoa.obscore 
      NATURAL JOIN  ivoa.obscore-radio
      WHERE dataproduct_type = 'visibilty' 
      AND s_largest_angular_scale_max > 0.018 
      AND uv_distribution_fill > 0.7 
      AND uv_distribution_ecc < 0.1  
      AND CONTAINS(POINT(s_ra,s_dec),CIRCLE(82.99458, 33.14794,0.0003)) =1))
\end{verbatim}

\subsection{Use case -  time selection }
\textit{Any timeseries obtained in beam forming mode with time resolution better than 0.1 second and time sample maximal duration smaller than 0.05 second. }\\ \\
Show me all datasets satisfying:\\
I. Instrument type = beam forming\\
II. Product type = time series  \\
III. time resolution < 0.1 s\\
IIV. maximal duration per sample  < 0.05 s \\

\begin{verbatim}
       SELECT * FROM ivoa.obscore 
      NATURAL JOIN  ivoa.obscore-radio
      WHERE instr_type = 'beamForming' 
      AND dataproduct_type = 'timeseries' 
      AND t_resolution < 0.1 AND
      AND t_exp_max < 0.05 
\end{verbatim}

\subsection{Use case -  instrument type and mode selection }
\label{sec:instrument-mode}
\textit{Any single dish  dataset  in raster mode around 3C 273.}\\ \\
Show me all datasets satisfying:\\
I. Instrument type is Single dish \\
II. Scan mode is raster \\
III. Target name = 3C 273 or \\
IV. position inside 3 arcmin around  	187.2779159404900 +02.0523882305500
\begin{verbatim}
      SELECT * FROM ivoa.obscore 
      NATURAL JOIN  ivoa.obscore-radio
      WHERE instr_type = 'SD' 
      AND scan_mode = 'raster' 
      AND (target_name = '3C 273' OR
       CONTAINS(POINT(s_ra,s_dec), CIRCLE(187.2779159404900, +02.0523882305500,0.05)) = 1)     
\end{verbatim}



\subsection{Use case - instrument type and frequency selection }
\textit{Any single dish or beam forming dataset  with spectral range inside the 500 Mhz - 1Ghz band and in the Coma Cluster.}\\ \\
Show me all datasets satisfying:\\
I. Instrument type is Single dish or beam forming \\
II. lower spectral limit > 500 Mhz \\
III. higher spectral limit < 1Ghz \\
IV. position inside 18 arcmin around 194.93502 +27.91246

\begin{verbatim}
      SELECT * FROM ivoa.obscore 
      NATURAL JOIN  ivoa.obscore-radio
      WHERE (instr_type = 'SD' OR
      instr_type = 'beamForming') 
      AND 299792458 / em_max > 500 
      AND 299792458 / em_min < 1000 
      AND CONTAINS(POINT(s_ra,s_dec),CIRCLE(194.93502, +27.91246, 0.3) = 1     
\end{verbatim}



\subsection{Use case - instrument parameters selection }
\textit{Any interferometry data of good quality and significant spatial resolution from the instrumental point of view. }\\ \\
Show me all datasets satisfying:\\
I. Instrument type = visibility \\
II. number of antennae > 40 \\
III. antenna maximum distance in meters  > 5000 \\

\begin{verbatim}
      SELECT * FROM ivoa.obscore-radio
      WHERE instr_type = 'interferometry' 
      AND instr_tel_number > 40 
      AND instr_tel_max_dist > 5000 
\end{verbatim}






\bibliography{ivoatex/ivoabib,ivoatex/docrepo}

\end{document}
